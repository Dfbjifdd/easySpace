\documentclass[11pt,a4paper]{report}
\usepackage[utf8]{inputenc}
\usepackage[russian]{babel}
\usepackage[OT1]{fontenc}
\usepackage{amsmath}
\usepackage{amsfonts}
\usepackage{amssymb}
\usepackage{graphicx}
\author{Григорий Субботин, Кирилл Муратов, Света Горбунова}
\title{Документация по проекту easySpace}
\begin{document}

\begin{titlepage}
 \begin{center}
    \large
    МИНИСТЕРСТВО ОБРАЗОВАНИЯ И НАУКИ\\ РОССИЙСКОЙ ФЕДЕРАЦИИ
     
    \textit{ФГБОУ ВО Российской Федерации}
    \vspace{0.5cm}
 
    АЛТАЙСКИЙ ГОСУДАРСТВЕННЫЙ УНИВЕРСИТЕТ
    
    \vspace{0.25cm}
     
    \textit{Институт цифровых технологий, электроники и физики}
    
    \vfill 
    Информатика и вычислительная техника
    \vfill
    \textsc{Групповая работа}\\[5mm]
     
    {\LARGE «Разработка графической программы с использованием библиотеки Tkinter, моделирующую солнечную систему»}
  \bigskip
     
     2 курс, группа 595
\end{center}
\vfill
 
\newlength{\ML}
\settowidth{\ML}{«\underline{\hspace{0.6cm}}» \underline{\hspace{1cm}}}
\hfill\begin{minipage}{0.4\textwidth}
  Руководитель курса\\
  \underline{\hspace{\ML}} И.\,А.~Шмаков\\
  «\underline{\hspace{0.5cm}}» \underline{\hspace{1cm}} 2020 г.
\end{minipage}%
\bigskip
 
\hfill\begin{minipage}{0.4\textwidth}
  Студенческая группа\\
  \underline{\hspace{\ML}} Г.\,М.~Субботин - менеджер проекта\\
  \underline{\hspace{\ML}} К.\,М.~Муратов - программист\\
  \underline{\hspace{\ML}} С.\,М.~Горбунова - тестировщик\\
  «\underline{\hspace{0.5cm}}» \underline{\hspace{1cm}} 2020 г.
\end{minipage}%
\vfill
 
\begin{center}
  Барнаул, 2020 г.
\end{center}
\end{titlepage}

\tableofcontents
\newpage
\section{Теория}

\subsection{Выбранный язык программирования}
Основным языком программирования (далее ЯП) стал Python. 
\section{Постановка задачи}


\section{Цель проекта}


\section{Описание классов}

\bibliography{Список литературы}



\section{Приложение 1}

\section{Приложение 2}

\end{document}
